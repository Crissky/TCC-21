\section{Introdução}
A Intelivix é uma empresa B2B (\textit{business-to-business} ou "de empresa para empresa", isto é, empresas que têm como clientes outras empresas) que oferece suporte jurídico para outras empresas usando diversas tecnologias da informação. O seu objetivo, em suas próprios palavras, é disponibilizar "inteligência artificial para gestão de risco jurídico" \cite{intelivix:2021}. A companhia trata de todas as etapas desse processo, desde a obtenção dos dados dos processos nos sites dos tribunais em todo o Brasil, passando pela "limpeza" e estruturação desses dados, até a disponibilização das informações, obtidas por meio desses dados, em forma de painéis informativos.

\subsection{Contexto/Problema}
Muitas empresas que possuem uma grande quantidade de clientes e tem uma operação que abrange uma ampla área podem ter problemas para gerenciar uma grande quantidade de processos judiciais. Muitas vezes pode não compensar ter uma equipe jurídica do tamanho necessário para lidar com o volume de processos que são iniciados contra a empresa, devido aos grandes gastos que podem prejudicar a empresa financeiramente. Além disso, existe a dificuldade de lidar com a grande massa de ações judiciais dos mais diversos tipos, que avançam em ritmos distintos e que podem estar em diversas comarcas.
 
% Em qual contexto seu trabalho está inserido? 
% Quais lacunas são observadas nesse contexto?
% Qual a motivação/justificativa do trabalho? (porque fechar essas lacunas é importante?)

\subsection{Objetivos}
Diante desse problema, o objetivo é capturar e armazenar continuamente os dados dos processos judiciais o mais breve possível na própria fonte, os sites dos tribunais, e, da mesma maneira, as suas atualizações.

Os dados que foram capturados dos sites tem diversas origens e, com isso, diversos formatos, isto é, não seguem uma mesma convenção. Então, após isso, é necessário realizar a limpeza desses dados e uma padronização para facilitar as manipulações posteriores. Depois de limpos, os dados serão extraídos, transformados e, a partir deles, serão geradas as mais diversas informações sobre os processos judicias. É nesse ponto dos procedimentos que este artigo irá se debruçar.

Por fim, uma vez transformadas, as informações serão cruzadas e relacionadas para serem apresentadas de uma maneira que facilite no direcionamento dos departamentos jurídicos, além de contribuir para a tomada de decisão dos gestores jurídicos.

% Qual o objetivo geral do trabalho?
% Quais são os objetivos específicos?

\subsection{Abordagem proposta}

\subsection{Contribuições}

\subsection{Organização do trabalho}

% Como o trabalho está organizado?