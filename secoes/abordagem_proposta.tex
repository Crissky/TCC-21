\section{Abordagem proposta}
% Nessa seção devem ser apresentados os métodos utilizados no trabalho, ferramentas, dados e materiais.

Empresas que atuam em uma grande área do território nacional atendendo a um grande número de clientes podem ser ré de diversas ações judiciais iniciadas pelos consumidores que se sentiram lesados com os serviços prestados. Desse modo, essas numerosas ações judiciais podem ocorrer em diferentes locais ou comarcas, \enquote{\textit{território em que o juiz de primeiro grau irá exercer sua jurisdição e pode abranger um ou mais municípios}} \cite{cnj:comarca}, o que dificulta o acompanhamento dos processos forenses. Assim sendo, manter uma equipe jurídica do tamanho necessário para cuidar dessa volumosa soma de processos, manualmente, pode ser muito custoso para a empresa, sendo capaz de prejudicar á saúde financeira e jurídica da dela.

Diante desse cenário, surgiram empresas que ofertam \enquote{inteligência jurídica}, acompanhando os processos jurídicos por meio da coleta dos dados destes processos, que são disponibilizados por meio das plataformas digitais providenciadas pelos sistemas de justiça. Essa coleta otimiza o trabalho das equipes jurídicas das empresas. No entanto, esses dados podem ser usados para fornecer muito mais informações para a empresa, desde de que sejam devidamente tratados e estruturados.

Mas no Brasil existem diversas plataformas digitais de justiça, em que cada uma disponibiliza os dados a sua maneira. Essa diferenciação na representação de um mesmo tipo de informação é uma problema quando se deseja usar técnicas de \textit{business intelligence} (BI), pois sem um tratamento prévio, o agrupamento dos dados dessa maneira não apresenta resultados satisfatórios.