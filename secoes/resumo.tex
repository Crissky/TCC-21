\begin{resumo}
    Grandes empresas podem ter dificuldades em gerenciar numerosos processos judiciais em que são rés, pois uma equipe do tamanho necessário para gerir esses processos pode ser muito dispendiosa. Com a digitalização dos processos judiciais, uma estratégia que automatize a organização das informações processuais e que seja de fácil consulta pode ser muito proveitosa. Diante disso, o data warehouse parece ser uma técnica muito indicada para tal, por normalizar e padronizar as informações de diversas fontes que estão contidas nele. Mas para carregar os dados em um data warehouse, é necessário prepará-los com um método comumente conhecido por ETL. Com esse método podemos criar fluxos de extração, transformação e carga de dados que serão escritos no data warehouse. Com um data warehouse preenchido com dados judiciais, podemos utilizá-lo para auxiliar a tomada de decisão do setor jurídico das empresas por meio da jurimetria, que consiste na aplicação de ferramentas estatísticas ao Direito, fornecendo novos pontos de vista baseado em dados.

    \hfill

    \textbf{Palavras-chaves}: ETL, Data Warehouse, SQL, Python, Jurimatria

%   Este trabalho descreve o fluxo do tratamento de dados jurídicos, obtidos de diversos sistemas digitais de justiça, com o objetivo de aplicar técnicas de ETL para o enriquecimento desses dados e para a construção de um \textit{data warehouse} com foco em jurimetria e que auxiliará os \textit{stakeholders} na tomada de decisão.
\end{resumo}

\begin{abstract}
  Large companies may have difficulty managing numerous court cases they are defendants in, as a team of the size needed to manage these cases can be very costly. With the digitization of judicial processes, a strategy that automates the adjustment of process information and is easy to consult can be very useful. Therefore, the data warehouse seems to be a very suitable technique for this, as it normalizes and standardizes information from different sources that are contained in it. But to load the data into a data warehouse, it's necessary to prepare it with a method commonly known as ETL. With this method we can create data extraction, transformation and load flows that will be written in the data warehouse. With a data warehouse filled with legal data, we can use it to help decision-making in the legal sector of companies through jurimetry, which consists of applying statistical tools to Law, providing new points of view based on data.
  
  \hfill

    \textbf{Keywords}: ETL, Data Warehouse, SQL, Python, Jurismetry

\end{abstract}