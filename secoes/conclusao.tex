\section{Conclusões}
\label{sec:conclusao}
% Relembrar qual era o objetivo do trabalho.
% Quais resultados foram alcançados?
% Quais conclusões podemos tirar a partir dos resultados?
% Quais os trabalhos futuros?

Lidar com um imenso volume de dados é um problema muito comum para as empresas na \enquote{era do \textit{Big Data}}. Ainda nesse sentido, a tendência é que essas informações crescem cada vez mais à medida que mais e mais sistemas são informatizados. Hoje, essa conjuntura já está presente na área jurídica.

Para tratar dessas questões, a estratégia da jurimetria aplicada em um \textit{data warehouse} se mostra bastante eficaz para superar essa enorme massa de dados, usando-os para extrair informações e novas perspectivas a respeito do ambiente em que se está inserido.

No entanto, uma etapa anterior ao \textit{data warehouse}, definida em três passos, é essencial para o seu êxito: a extração, a transformação e a carga desses dados deve padronizar e estruturar toda a porção de dados que irá compor o \textit{data warehouse}.

Dessa maneira, esse aglomerado de dados se torna valioso para o negócio, mostrando para a empresa seus pontos fortes e fracos, nesse caso no âmbito jurídico, guiando as suas ações para as áreas que necessitam de sua atenção e podendo contribuir com a antecipação de problemas futuros por meio de uma abordagem estatística.

Atualmente, a maior parte dos dados que são utilizados pelo projeto percorrem um longo caminho até o seu destino. Eles transitam desde o MongoDB, passando pelo processamento do Python e pela reestruturação do Athena, até desembocar no PostgreSQL. No entanto, o Athena é uma ferramenta \enquote{poderosa} o suficiente para realizar todas as operações necessárias para aplicar o ETL sobre o conjunto de dados, dispô-los em um formato colunar, como o Apache Parquet ou \textit{Optimized Row Columnar} (ORC), que são formatos mais rápidos para a pesquisa quando comparados aos formatos em linha, e, finalmente, servir como um banco de dados fim que fornecerá as informações que serão carregadas no \textit{data warehouse}. Destarte, essa seria uma ótima atualização para ser feita no futuro por diminuir o tempo de processamento total e por simplificar a metodologia emprega, já que diminuiria o número de sistemas que precisariam de manutenção.