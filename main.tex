%%%%%%%%%%%%%%%%%%%%%%%%%%%%%%%%%%%%%%%%%%%%%%%%%%%%%%%%%%%%%%%%%%%%%%
% How to use writeLaTeX: 
%
% You edit the source code here on the left, and the preview on the
% right shows you the result within a few seconds.
%
% Bookmark this page and share the URL with your co-authors. They can
% edit at the same time!
%
% You can upload figures, bibliographies, custom classes and
% styles using the files menu.
%
%%%%%%%%%%%%%%%%%%%%%%%%%%%%%%%%%%%%%%%%%%%%%%%%%%%%%%%%%%%%%%%%%%%%%%

\documentclass[12pt]{article}
\usepackage{amsmath,amssymb}

\usepackage{sbc-template}

\usepackage{graphicx,url}

\usepackage{indentfirst}

\usepackage[brazil]{babel}   
\usepackage[utf8]{inputenc}  

\usepackage{csquotes}

\sloppy

\title{Título}

\author{Edson Fagner da Silva Cristovam \inst{1} }

\address{Departamento de Estatística e Informática -- Universidade Federal Rural de Pernambuco\\
  Rua Dom Manuel de Medeiros, s/n, - CEP: 52171-900  -- Recife -- PE -- Brasil
  \email{fagner.cristovam@hotmail.com}
}

\begin{document} 

\maketitle

% SEÇÕES
\begin{resumo} 
  ESCREVA SEU RESUMO AQUI
  \end{resumo}

\begin{abstract}
  [do write your abstract in english, use Google Translate and do corrections]
\end{abstract}
\section{Introdução}
A Intelivix é uma empresa B2B (\textit{business-to-business} ou "de empresa para empresa", isto é, empresas que têm como clientes outras empresas) que oferece suporte jurídico para outras empresas usando diversas tecnologias da informação. O seu objetivo, em suas próprios palavras, é disponibilizar "inteligência artificial para gestão de risco jurídico" \cite{intelivix:2021}. A companhia trata de todas as etapas desse processo, desde a obtenção dos dados dos processos nos sites dos tribunais em todo o Brasil, passando pela "limpeza" e estruturação desses dados, até a disponibilização das informações, obtidas por meio desses dados, em forma de painéis informativos.

\subsection{Contexto/Problema}
Muitas empresas que possuem uma grande quantidade de clientes e tem uma operação que abrange uma ampla área podem ter problemas para gerenciar uma grande quantidade de processos judiciais. Muitas vezes pode não compensar ter uma equipe jurídica do tamanho necessário para lidar com o volume de processos que são iniciados contra a empresa, devido aos grandes gastos que podem prejudicar a empresa financeiramente. Além disso, existe a dificuldade de lidar com a grande massa de ações judiciais dos mais diversos tipos, que avançam em ritmos distintos e que podem estar em diversas comarcas.
 
% Em qual contexto seu trabalho está inserido? 
% Quais lacunas são observadas nesse contexto?
% Qual a motivação/justificativa do trabalho? (porque fechar essas lacunas é importante?)

\subsection{Objetivos}
Diante desse problema, o objetivo é capturar e armazenar continuamente os dados dos processos judiciais o mais breve possível na própria fonte, os sites dos tribunais, e, da mesma maneira, as suas atualizações.

Os dados que foram capturados dos sites tem diversas origens e, com isso, diversos formatos, isto é, não seguem uma mesma convenção. Então, após isso, é necessário realizar a limpeza desses dados e uma padronização para facilitar as manipulações posteriores. Depois de limpos, os dados serão extraídos, transformados e, a partir deles, serão geradas as mais diversas informações sobre os processos judicias. É nesse ponto dos procedimentos que este artigo irá se debruçar.

Por fim, uma vez transformadas, as informações serão cruzadas e relacionadas para serem apresentadas de uma maneira que facilite no direcionamento dos departamentos jurídicos, além de contribuir para a tomada de decisão dos gestores jurídicos.

% Qual o objetivo geral do trabalho?
% Quais são os objetivos específicos?

\subsection{Abordagem proposta}

\subsection{Contribuições}

\subsection{Organização do trabalho}

% Como o trabalho está organizado?
\section{Referencial teórico}
\label{sec:referencialTeorico}
% Os principais conceitos aos quais os trabalho está relacionado. Por exemplo, apresentar os conceitos de Aprendizado de Máquina e métodos relacionados.
% As principais técnicas e métodos nos quais o trabalho se apoia (referências)

Antes de tratar sobre as definições mais específicas que este trabalho vai utilizar, gostaria de trazer uma definição mais geral em que o problema está incluído. Podemos compreender Sistemas de Informação como sistemas, sejam eles manuais ou automatizados, inter-relacionados que atuam armazenando, processando e transmitindo dados que representam informações para as partes interessadas (\textit{stakeholders}). De forma mais geral, esses sistemas \enquote{\textit{atuam em conjunto para o cumprimento de uma tarefa ou um objetivo}} \cite{turban2009business}.

O grupo básico de operações de um Sistema de Informação é constituído pelas entradas, pelo processamento e pelas saídas (Figura \ref{fig:operacoesBasicaSistemas}). A entrada é o conjunto de dados que será usado pelo sistema, o processamento é composto pela transformações e combinações que os dados de entradas serão submetidos e, por fim, a saída é o produto obtido por meio do processamento dos dados de entrada. Em alguns casos, o sistema pode ser retroalimentado (\textit{feedback}), fazendo com que o resultado de saída seja manipulado como dados de entrada em uma nova execução. Nesse processo, também é comum que ocorra o armazenamento dos dados de entrada e de saída para serem utilizados posteriormente, como pode ser visto na Figura \ref{fig:operacoesBasicaSistemas}.

\begin{figure}[ht]
\centering
\includegraphics[width=1\textwidth]{imagens/operacoes-basicas-sistema-informacao.png}
\caption{Operações básicas de um Sistema de Informação.}
\label{fig:operacoesBasicaSistemas}
\end{figure}

% TODO falar sobre informação e dados https://integrada.minhabiblioteca.com.br/reader/books/9786556901916/pageid/17

Os dados que servem como entrada do sistema \enquote{\textit{são a menor parte de uma informação}} \cite{boscarioli2016mineracao} e podem ser definidos como conhecimento bruto ou fatos isolados. Nesse estado, podemos tê-los como não adequadamente tratados para fornecer gnose aos \textit{stakeholders}.

Por sua vez, a informação é a saída do sistema e descreve \enquote{\textit{qualquer conhecimento do mundo real... mas que apresenta algum significado ou valor para quem o detém}} \cite{boscarioli2016mineracao}. A informação é formada pelos dados de entrada do sistema após serem transformados, combinados \enquote{\textit{relacionados logicamente e organizados para atingir um resultado definido}} \cite{vida2021datawarehouse}, passando a fornecer conhecimento relevante para o processo de tomada de decisão.

Dessa maneira, temos os dados como matéria-prima do sistema e a informação como o produto. O nome de uma pessoa, uma data e um valor monetário são bons exemplos de dados, isolados eles não significam muito. No entanto, quando combinamos esses dados e atribuímos um significado ou contexto, como um depósito bancário, eles, os dados, passam a ter valor e se tornam informação.
\section{Ferramentas Utilizadas}
Discorrer sobre as ferramentas que são utilizadas para efetuar o trabalho.
\section{Abordagem proposta}
% Nessa seção devem ser apresentados os métodos utilizados no trabalho, ferramentas, dados e materiais.

Empresas que atuam em uma grande área do território nacional atendendo a um grande número de clientes podem ser ré de diversas ações judiciais iniciadas pelos consumidores que se sentiram lesados com os serviços prestados. Desse modo, essas numerosas ações judiciais podem ocorrer em diferentes locais ou comarcas, \enquote{\textit{território em que o juiz de primeiro grau irá exercer sua jurisdição e pode abranger um ou mais municípios}} \cite{cnj:comarca}, o que dificulta o acompanhamento dos processos forenses. Assim sendo, manter uma equipe jurídica do tamanho necessário para cuidar dessa volumosa soma de processos, manualmente, pode ser muito custoso para a empresa, sendo capaz de prejudicar a bem-estar financeiro e jurídico dela.

Diante desse cenário, surgiram empresas que ofertam \enquote{inteligência jurídica}, acompanhando os inúmeros processos jurídicos por meio da sua coleta, armazenando-os em um banco de dados. Esses dados processuais são disponibilizados pelos sistemas de justiça providenciados por meio das suas plataformas digitais como o Projudi do TJPE \cite{tjpe} e o e-SAJ CE \cite{esajce}. Essa coleta otimiza o trabalho das equipes jurídicas das empresas, no entanto, esses dados podem ser usados para fornecer muito mais informações para o setor jurídico da empresa e até contribuir para o seu \textit{business intelligence} (BI), revelando novas perspectivas que servirão de base para a tomada de decisão.

Mas no Brasil existem diversas plataformas digitais de justiça, em que cada uma disponibiliza os dados a sua maneira. Essa diferenciação na representação de um mesmo tipo de informação é uma problema quando se deseja usar técnicas de \textit{business intelligence} por meio de um \textit{data warehouse} (Seção \ref{subsec:datawarehouse}), pois sem um tratamento prévio, o agrupamento dos dados nessa situação não apresenta resultados satisfatórios. Então é necessário utilizar alguma estratégia para padronizar esses dados, sendo o ETL (Seção \ref{subsec:etl}) uma das técnicas mais utilizadas para tal.

\subsection{Processos}
\label{processos}

As elementos contidos no processo são imprescindíveis para a construção do \textit{data warehouse} focado na obtenção de conhecimento jurídico e na aplicação da jurimetria, isto é, a \enquote{\textit{é a estatística aplicada ao Direito}} \cite{newlawJurimetria}.

Os principais dados de um processo judicial são a sua Numeração Processual Única (NPU) ou somente número, as partes envolvidas e os andamentos ou movimentações.

A NPU é composta por vinte dígitos e foi instituído pelo Conselho Nacional de Justiça (CNJ) com o intuito de facilitar a consulta das informação referente a um processo, pois os números dos processos mudavam em cada instância ou recurso. A Numeração Processual Única vale para os tribunais de todo o país e apresentam a seguinte estrutura \enquote{NNNNNNN-DD.AAAA.J.TR.OOOO} no qual \cite{jusbrasilNPU}:

\begin{itemize}
    \item \textbf{NNNNNNN-DD}: representam o número sequencial do processo e seu dígito verificador \cite{TRF4NPU}.
    \item \textbf{AAAA}: diz respeito ao ano de avaliação do processo \cite{TRF4NPU}.
    \item \textbf{J}: informa qual o órgão ou segmento do Poder Judiciário que o processo pertence \cite{TRF4NPU}.
    \item \textbf{TR}: indica o tribunal do respectivo segmento ou circunscrição judiciária \cite{TRF4NPU}.
    \item \textbf{OOOO}: remete à unidade de origem do processo \cite{TRF4NPU}.
\end{itemize}

As partes do processo são compostas pelo autor, que desempenha o papel de polo ativo do processo, ou seja, aquele que recorre a tutela jurídica do estado, tomando a posição ativa, e pelo réu, que realiza o papel do polo passivo, estando sujeito a ação processual iniciada pelo autor.

Por fim, as movimentações de um processo são um conjunto de informações que comunicam a evolução processual ao longo da tramitação. Elas podem conter os mais diferentes tipos de conhecimentos a respeito do processo, como os informes sobre o dia de uma audiência, notificação de emissão de intimação ou até mesmo a notícia da sentença do juiz.

\subsection{ETL para Jurimetria}
\label{jurimetria}

A técnica de ETL usada neste trabalho segue o modelo de filas FIFO (\textit{First In, First Out} - primeiro a entrar, primeiro a sair) (Figura \ref{fig:stepsFlow}). Cada nó dessa fila é uma estrutura que recebe o nome de \textit{step}.

\begin{figure}[ht]
\centering
\includegraphics[width=1\textwidth]{imagens/steps-flow.png}
\caption{Exemplo do fluxo dos \textit{steps}.}
\label{fig:stepsFlow}
\end{figure}

O \textit{step} é um conjunto de operações que serão aplicada ao processo para realizar um objetivo bem definido. No projeto, ele é codificado como uma classe Python (Seção \ref{subsec:python}) com um método obrigatório chamado \textit{transform}. O \textit{transform} é a função que receberá o processo e aplicará as transformações e inferências necessárias para atingir o seu objetivo.

\begin{figure}[ht]
\centering
\includegraphics[width=1\textwidth]{imagens/processo-in-step.png}
\caption{Exemplo de \textit{step} enriquecendo o documento com novas informações.}
\label{fig:processoInStep}
\end{figure}

Os processos são extraídos dos vários sistemas jurídicos pela equipe de extração e armazenados como documentos em um banco de dados MongoDB (Seção \ref{subsec:mongo}). Esses documentos serão usados no processamento do ETL que preparará os dados para serem inseridos no \textit{data warehouse}. No ETL, os documentos são lidos do MongoDB e estruturados como um dicionário do Python (Figura ).

% TODO inserir imagem do processo como um dicionário python

\subsubsection{Steps}
\label{steps}

Para esse trabalho, não serão abordados todos os \textit{steps} que são efetivamente realizados na empresa, mas somente um recorte de alguns dos mais importantes. Com exceção do primeiro \textit{step}, que trata da normalização dos dados, cada um dos demais \textit{steps} são responsáveis pela criação de uma nova tabela no banco de dados final.

% TODO Escreve sobre os steps.
O primeiro \textit{step} é chamado de...
\section{Conclusões}
\label{sec:conclusao}
% Relembrar qual era o objetivo do trabalho.
% Quais resultados foram alcançados?
% Quais conclusões podemos tirar a partir dos resultados?
% Quais os trabalhos futuros?

Lidar com um imenso volume de dados é um problema muito comum para as empresas na \enquote{era do \textit{Big Data}}. Ainda nesse sentido, a tendência é que essas informações crescem cada vez mais à medida que mais e mais sistemas são informatizados. Hoje, essa conjuntura já está presente na área jurídica.

Para tratar dessas questões, a estratégia da jurimetria aplicada em um \textit{data warehouse} se mostra bastante eficaz para superar essa enorme massa de dados, usando-os para extrair informações e novas perspectivas a respeito do ambiente em que se está inserido.

No entanto, uma etapa anterior ao \textit{data warehouse}, definida em três passos, é essencial para o seu êxito: a extração, a transformação e a carga desses dados deve padronizar e estruturar toda a porção de dados que irá compor o \textit{data warehouse}.

Dessa maneira, esse aglomerado de dados se torna valioso para o negócio, mostrando para a empresa seus pontos fortes e fracos, nesse caso no âmbito jurídico, guiando as suas ações para as áreas que necessitam de sua atenção e podendo contribuir com a antecipação de problemas futuros por meio de uma abordagem estatística.

Atualmente, a maior parte dos dados que são utilizados pelo projeto percorrem um longo caminho até o seu destino. Eles transitam desde o MongoDB, passando pelo processamento do Python e pela reestruturação do Athena, até desembocar no PostgreSQL. No entanto, o Athena é uma ferramenta \enquote{poderosa} o suficiente para realizar todas as operações necessárias para aplicar o ETL sobre o conjunto de dados, dispô-los em um formato colunar, como o Apache Parquet ou \textit{Optimized Row Columnar} (ORC), que são formatos mais rápidos para a pesquisa quando comparados aos formatos em linha, e, finalmente, servir como um banco de dados fim que fornecerá as informações que serão carregadas no \textit{data warehouse}. Destarte, essa seria uma ótima atualização para ser feita no futuro por diminuir o tempo de processamento total e por simplificar a metodologia emprega, já que diminuiria o número de sistemas que precisariam de manutenção.

% TODO
% Introdução - Contexto: Falar com Rômero sobre os problemas que os clientes buscam resolver.
% GERAL: Falar com a professora Roberta sobre o trabalho e pedir sugestões e indicação de uma livro sobre ETL.

% Referências
% As referências são inseridas no arquivo  sbc-template.bib. Neste arquivo estão descritos os tipos mais comuns de referência.

Para citar referência use \cite{knuth:84}, \cite{boulic:91},  and \cite{smith:99}.

\bibliographystyle{sbc}
\bibliography{sbc-template}

\end{document}
