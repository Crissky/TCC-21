\section{Abordagem proposta}
% Nessa seção devem ser apresentados os métodos utilizados no trabalho, ferramentas, dados e materiais.

Empresas que atuam em uma grande área do território nacional atendendo a um grande número de clientes podem ser ré de diversas ações judiciais iniciadas pelos consumidores que se sentiram lesados com os serviços prestados. Desse modo, essas numerosas ações judiciais podem ocorrer em diferentes locais ou comarcas, \enquote{\textit{território em que o juiz de primeiro grau irá exercer sua jurisdição e pode abranger um ou mais municípios}} \cite{cnj:comarca}, o que dificulta o acompanhamento dos processos forenses. Assim sendo, manter uma equipe jurídica do tamanho necessário para cuidar dessa volumosa soma de processos, manualmente, pode ser muito custoso para a empresa, sendo capaz de prejudicar a bem-estar financeiro e jurídico dela.

Diante desse cenário, surgiram empresas que ofertam \enquote{inteligência jurídica}, acompanhando os inúmeros processos jurídicos por meio da sua coleta, armazenando-os em um banco de dados. Esses dados processuais são disponibilizados pelos sistemas de justiça providenciados por meio das suas plataformas digitais como o Projudi do TJPE \cite{tjpe} e o e-SAJ CE \cite{esajce}. Essa coleta otimiza o trabalho das equipes jurídicas das empresas, no entanto, esses dados podem ser usados para fornecer muito mais informações para o setor jurídico da empresa e até contribuir para o seu \textit{business intelligence} (BI), revelando novas perspectivas que servirão de base para a tomada de decisão.

Mas no Brasil existem diversas plataformas digitais de justiça, em que cada uma disponibiliza os dados a sua maneira. Essa diferenciação na representação de um mesmo tipo de informação é uma problema quando se deseja usar técnicas de \textit{business intelligence} por meio de um \textit{data warehouse} (Seção \ref{subsec:datawarehouse}), pois sem um tratamento prévio, o agrupamento dos dados nessa situação não apresenta resultados satisfatórios. Então é necessário utilizar alguma estratégia para padronizar esses dados, sendo o ETL (Seção \ref{subsec:etl}) uma das técnicas mais utilizadas para tal.