\section{Ferramentas Utilizadas}
\label{sec:ferramentas}
Esta seção irá tratar das principais ferramentas utilizadas no projeto para a execução de todas as etapas do ETL, desde a extração dos dados, passando pela transformação e finalizando com a sua carga.

\subsection{Python}
\label{subsec:python}
Python é uma linguagem de programação desenvolvida no final dos anos 80. Ela foi concebida pelo matemático holandês, Guido van Rossum, que a criou quando sentiu a necessidade de uma linguagem mais simples de se trabalhar que a C e que não tivesse as limitações do \textit{shell script}, linguagem de comando utilizada por ele \cite{de2020python}. Segundo pesquisa do Stack Overflow, um site de perguntas e respostas para programadores, Python é a terceira linguagem mais popular na plataforma \cite{stackoverflow:survey}. Ela é bastante conhecida no seu uso em automação de tarefas, desenvolvimento web e aprendizado de máquina devido à sua sintaxe sucinta e breviloquente \cite{de2020python}.

Python é uma linguagem de alto nível, isto significa que, teoricamente, ela está mais próxima da linguagem natural que da linguagem de máquina. Também é uma linguagem interpretada, na qual o código fonte é convertido para o formato intermediário chamado de \textit{bytecode} e o interpretador fica responsável por converter o \textit{bytecode} para o formato próprio para ser executado no sistema operacional desejado. Python é uma linguagem de programação multiparadigma, que se encaixa em muitos paradigmas como o imperativo e funcional. A tipagem do Python é dinâmica, pois a linguagem determina o tipo de dado de uma variável baseada no valor que essa variável armazenará, não havendo necessidade de declarar o tipo explicitamente \cite{de2020python}.


\subsection{Amazon Web Services}
\label{subsec:aws}

% https://aws.amazon.com/pt/
% https://aws.amazon.com/pt/what-is-aws/?nc1=f_cc

Amazon Web Services, ou simplesmente AWS, é a plataforma de nuvem da Amazon que disponibiliza mais de 200 serviços de \textit{datacenters} ao redor do globo. A plataforma oferece infraestrutura em nuvem como computação, armazenamento de arquivos ou em banco de dados, criptografia, ferramentas de \textit{machine learning}, \textit{data lakes} e \textit{internet} das coisas, além de suporte ao Docker e às linguagens de programação: .NET, Python, Java, JavaScript e PHP. A AWS dispõe de uma infraestrutura que satisfaz requisitos de segurança militares, compatível com 90 normas de segurança e certificações. Sua rede oferece alta redundância e taxa de transferência com baixa latência, estando presente em 81 zonas de disponibilidade que estão distribuídas em 25 regiões geográficas ao redor de globo \cite{aws:about}.

\subsubsection{AWS Athena}
\label{subsec:athena}
% https://aws.amazon.com/pt/athena/

O AWS Athena, ou somente Athena, é um serviço de consultas de dados \textit{serverless}, isto é, que segue um modelo de execução de computação em nuvem alocando recursos de processamento sob demanda e que elimina a necessidade do usuário de gerenciar a infraestrutura do servidor. O serviço é baseado no \textit{software} de código aberto chamado Presto, da The Presto Foundation, ele usa o padrão SQL (\textit{Structured Query Language} ou Linguagem de Consulta Estruturada) para acessar os dados dos arquivos que ficam armazenados no Amazon S3 (Seção \ref{subsec:s3}). É possível configurar o Athena para que ele trate um ou mais conjuntos de arquivos de forma análoga ao modelo relacional, estruturando-os em forma de tabelas e atribuindo tipos para as colunas, isso graças à sua integração com o AWS Glue Data Catalog. O AWS Athena tem suporte para arquivos organizados em linhas como CSV, Avro e JSON e os formatos colunares ORC e Parquet \cite{aws:athena}.

\subsubsection{Amazon S3}
\label{subsec:s3}
% https://aws.amazon.com/pt/s3/

Amazon Simple Storage Service (Amazon S3) é um serviço da AWS para armazenamento de arquivos de forma escalável. Os dados armazenados no S3 podem ser usados em diversos casos de uso como \textit{data lakes}, \textit{sites} e aplicações móveis. O serviço também oferece suporte à criptografia e ao gerenciamento de autorizações, permitindo que a organização administre o nível de acesso dos seus colaboradores. Ele fornece uma durabilidade para os arquivos de 99,99\%, pois realiza cópias automáticas de todos os documentos e os conserva em diversas regiões, aumentando sua disponibilidade e protegendo-os contra erros ou falhas \cite{aws:s3}.


\subsection{MongoDB}
\label{subsec:mongo}
% https://www.mongodb.com/
O MongoDB é um \textit{software} de banco de dados orientado a documentos com o código aberto, sendo classificado como um NoSQL, também conhecido como \textit{Not only SQL} (Não somente SQL) – um termo usado para categorizar bancos de dados não relacionais, isto é, uma classe de bancos de dados que armazena e recupera as informações que são estruturadas de uma maneira diferente à da abordagem com relações tabulares. Ele é escrito na linguagem C++ e é distribuído em múltiplas plataformas. A organização dos documentos é feita em um formato similar ao JSON, em uma disposição composta por um par de campo e valor, em que os valores dos campos podem conter dados de tipos primitivos, outros documentos ou vetores, o que fornece a possibilidade de se criar hierarquias complexas que podem ser indexadas, o que facilita a recuperação das informações. Os documentos são agrupados em coleções, uma perspectiva análoga às tabelas do modelo relacional \cite{mongoDB2021}.

O MongoDB apresenta suporte a buscas por campo, intervalo de valor, expressões regulares e também por funções JavaScript personalizadas \cite{mongoDB2021}. As agregações de dados do MongoDB podem ser feitas usando um \textit{pipeline} de agregação ou com o \textit{map-reduce}, que está depreciado e possui um desempenho pior que o do \textit{pipeline} \cite{mongoDBagregacao}. Além disso, ele é capaz de executar transção ACID – acrônimo em inglês para: \textit{Atomicity}, \textit{Consistency}, \textit{Isolation}, \textit{Durability}, ou em uma tradução livre, Atomicidade, Consistência, Isolamento e Durabilidade — que é um conjunto de propriedades para as transações em banco de dados \cite{mongoDBACID}.


\subsection{PostgreSQL}
\label{subsec:postgres}
% https://www.postgresql.org/
% https://en.wikipedia.org/wiki/PostgreSQL
PostgreSQL é um Sistema Gerenciador de Banco de Dados (SGBD) de código aberto baseado na estrutura objeto-relacional. Ele utiliza o padrão SQL, estando de acordo com 170 dos 179 requisitos obrigatórios da norma, combinado com recursos que escalam as cargas de trabalho \cite{postgresAbout}.

Ele foi desenvolvido na Universidade da Califórnia em Berkeley como parte do projeto POSTGRES (1986) com liderança do Professor Michael Stonebraker e patrocínio da \textit{Defense Advanced Research Projects Agency} (DARPA), do \textit{Army Research Office} (ARO), da \textit{National Science Foundation} (NSF) e do ESL Inc. Somente em 1996 recebeu o nome que é usando até o momento, PostgreSQL, iniciando na versão 6, pois levou em consideração as versões desde o projeto original \cite{postgresHistory}.

O PostgresSQL é compatível com transações ACID e oferece suporte a dados geoespaciais com o \enquote{PostGIS}, a documentos JSON e XML, a funções personalizadas e a linguagens procedurais como: PL/PGSQL, Perl, Python. Com ele também é possível realizar consultas paralelizadas, particionamento de tabelas, indexação com \textit{b-tree}, \textit{multicolumn}, \textit{expressions} ou \textit{partial} e expressar caminhos no formato SQL/JSON. Além disso, sobre segurança, o PostgresSQL permite autenticação com GSSAPI, SSPI, LDAP, SCRAM-SHA-256 e certificado, bem como um sólido controle de acesso. Por fim, o PostgresSQL é altamente extensível, permitindo que o usuário possa construir soluções usando o próprio PostgresSQL como base \cite{postgresAbout}.